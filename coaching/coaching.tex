\documentclass{article}

\usepackage[a4paper,margin=1in]{geometry}
\usepackage[T1]{fontenc}
\usepackage{tgcursor}

\begin{document}
\fontfamily{qcr}\selectfont
\title{Interview Questions}
\author{Karl Solomon}

\maketitle
\small
\begin{itemize}
	\item Vince/AMZN:
	      \begin{itemize}
		      \item 12 Leadership Principles
				  \begin{enumerate}
				  	\item Customer Obsession \\
						Refusal to cut scope at SYK. Just put in more hours to get the job done on time and with sales' desired features.
				  	\item Ownership \\
						I was quite literally the software owner at SYK. I did everything from software architecture, component selection, driver development, wrote scheduler/dispatcher, in-field upgrade (USB).
						Cared highly about finding bugs fast at Nvidia. Integrated coverity, gcov, and unit tests into the RISCV Core RM. This identified numerous bugs and code bloat prior to Ada/Hopper tapeout.
				  	\item Invent and Simplify \\
				  	\item Be right often \\ 

				  	\item Learn and be curious \\
						Currently exploring some of the more recent c++ features (ranges, modules, in order to improve my ability to write cleaner and more efficient code.)
				  	\item Hire and develop the best
				  	\item Insist on highest standards \\
				  	\item Think big
				  	\item Bias for action \\
				  	\item Frugality
				  	\item Earn Trust \\
						X-Functional team's trust to be fully responsible for the software. They had confidence that any bug we identified I could resolve.
						Also Amimon's trust. SYK had a very strong-arm relationship with their clients, however I understood that we were making a lot of unrealistic demands of them. They understood that I could help translate what requests were reasonable and which were not.
				  	\item Dive deep \\
						Autofocus 
				  \end{enumerate}
		      \item Make sure to ask questions
		      \item Prioritize working solution. But very much interested in optimal approach. Try to identify best data structures/algorithms.
		      \item Make sure to discuss WHY I'm making specific decisions in my implementation.
		      \item Many recent ocus on concurrency/multithreading in low-level
		      \item virtual white-board. Will not actually be running code
		      \item 
		      \item 
		      \item 
		      \item 
		      \item 
		      \item https://amazon.jobs/content/en/how-we-hire/sde-ii-interview-prep
\end{itemize}
\end{itemize}
\end{document}
\enddocument


Concurrency and Multithreading:

Basic concepts of threads and processes
Thread synchronization mechanisms (mutexes, semaphores, locks)
Race conditions and deadlocks
Atomic operations
Thread-Safe Data Structures:

Concurrent collections (e.g., ConcurrentQueue, ConcurrentBag)
Lock-free data structures
Understanding the differences between thread-safe and non-thread-safe collections
Design Patterns for Concurrency:

Producer-Consumer pattern
Readers-Writer pattern
Thread pool pattern
Language-Specific Concurrency Features:

For Java: java.util.concurrent package
For C#: Task Parallel Library (TPL) and async/await
For C++: std::thread and <atomic> library
Callback Mechanisms:

Function pointers
Delegates (in languages that support them)
Lambda expressions
Performance Considerations:

Understanding the overhead of different synchronization mechanisms
Balancing thread safety with performance
Testing Multithreaded Code:

Techniques for writing unit tests for concurrent code
Tools for detecting race conditions and deadlocks
Distributed Systems Concepts:

While not directly related to this problem, understanding concepts like eventual consistency and distributed locking can be beneficial
Algorithms for Concurrent Operations:

Compare-And-Swap (CAS) operations
Lock-free algorithms
Memory Models:

Understanding memory barriers and volatile variables
Cache coherence issues in multi-core systems
Practice Problems:

Implement a thread-safe singleton
Create a simple producer-consumer queue
Implement a basic thread pool
Solve classic concurrency problems like the dining philosophers problem
Remember, for interviews, it's not just about knowing the solutions, but also being able to explain your reasoning, discuss trade-offs, and analyze the performance and correctness of your solutions.

Lastly, be prepared to write code on a whiteboard or in a simple text editor. Practice implementing these concepts without relying on an IDE's features.

List out all of the major problems I've had to overcome.
Be able to talk about each of those stories
PROBLEM
ACTION STEPS to solve
OUTCOME

for phone interview
most often:
deliver results, then ownership, then be curious


Good Framework to answer the behavioral
Solving problems which positively impact your team, company or external customers
•	A time when you weren’t satisfied with the status quo and made improvements
•	Incorporating feedback from a customer to improve a product or service
•	Taking initiative and ownership of problems which fall outside the scope your day to day duties 
•	Acquiring new skills in order to solve a problem
•	Figuring out solutions to ambiguous situations (data driven, dive deep, learn and be curious)
•	Learning from past mistakes or outcome which weren’t ideal
•	Delivering an important project under a tight deadline
•	What was the scope or impact of across the organization or customer base (the broader the impact the better)


Good example to write a few test cases in addition to the solution


