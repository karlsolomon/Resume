\documentclass{article}

\usepackage[a4paper,margin=1in]{geometry}
\usepackage[T1]{fontenc}
\usepackage{tgcursor}
\usepackage{hyperref}

\begin{document}
\fontfamily{qcr}\selectfont
\title{Interview Questions}
\author{Karl Solomon}

\maketitle
\small

\section{AMZN}
\subsection{Behavioral Coaching}
most often seen for this role: deliver results, then ownership, then be curious
\begin{itemize}
    \item Behavioral: list of toughest challenges faced on the job
	\item Good Framework to answer the behavioral
		\begin{itemize}
			\item Solving problems which positively impact your team, company or external customers
			\item A time when you weren’t satisfied with the status quo and made improvements
			\item Incorporating feedback from a customer to improve a product or service
			\item Taking initiative and ownership of problems which fall outside the scope your day to day duties 
			\item Acquiring new skills in order to solve a problem
			\item Figuring out solutions to ambiguous situations (data driven, dive deep, learn and be curious)
			\item Learning from past mistakes or outcome which weren’t ideal
			\item Delivering an important project under a tight deadline
			\item What was the scope or impact of across the organization or customer base (the broader the impact the better)
		\end{itemize}
	\item 12 Leadership Principles:
		\begin{enumerate}
			\item \textbf{Customer Obsession }\\
				Refusal to cut scope at SYK. Just put in more hours to get the job done on time and with sales' desired features.
			\item \textbf{Ownership }\\
				I was quite literally the software owner at SYK. I did everything from software architecture, component selection, driver development, wrote scheduler/dispatcher, in-field upgrade (USB).
				Cared highly about finding bugs fast at Nvidia. Integrated coverity, gcov, and unit tests into the RISCV Core RM. This identified numerous bugs and code bloat prior to Ada/Hopper tapeout.
			\item \textbf{Invent and Simplify }\\
			\item \textbf{Be right often }\\ 
			
			\item \textbf{Learn and be curious }\\
				Currently exploring some of the more recent c++ features (ranges, modules, in order to improve my ability to write cleaner and more efficient code.)
			\item \textbf{Hire and develop the best}
			\item \textbf{Insist on highest standards }\\
			\item \textbf{Think big}
			\item \textbf{Bias for action }\\
			\item \textbf{Frugality}
			\item \textbf{Earn Trust }\\
				X-Functional team's trust to be fully responsible for the software. They had confidence that any bug we identified I could resolve.
				Also Amimon's trust. SYK had a very strong-arm relationship with their clients, however I understood that we were making a lot of unrealistic demands of them. They understood that I could help translate what requests were reasonable and which were not.
			\item \textbf{Dive deep }\\
			Autofocus 
		\end{enumerate}
\end{itemize}
\subsection{Technical Coaching}
\href{https://amazon.jobs/content/en/how-we-hire/sde-ii-interview-prep}{See HERE for how they hire}
\begin{itemize}
	\item Generic Feedback/Setup
	\begin{itemize}
		\item Make sure to ask questions
		\item Prioritize working solution. But very much interested in optimal approach. Try to identify best data structures/algorithms.
		\item Make sure to discuss WHY I'm making specific decisions in my implementation.
		\item Many recent ocus on concurrency/multithreading in low-level
		\item Virtual white-board/shared text editor. Will not actually be running code
		\item See here for details: 
	\end{itemize}
	\item Specifics
		\begin{itemize}
			\item Concurrency and Multithreading \\
Basic concepts of threads and processes
Thread synchronization mechanisms (mutexes, semaphores, locks)
Race conditions and deadlocks
Atomic operations
			\item Thread-Safe Data Structures: \\
Concurrent collections
Concurrent collections (e.g., ConcurrentQueue, ConcurrentBag)
Lock-free data structures
Understanding the differences between thread-safe and non-thread-safe collections
			\item Design Patterns for Concurrency: \\
Producer-Consumer pattern
Readers-Writer pattern
Thread pool pattern
			\item Language-Specific Concurrency Features: \\
				For C++: std::thread and <atomic> library
			\item Callback Mechanisms: \\
Function pointers
Delegates (in languages that support them)
Lambda expressions
			\item Performance Considerations: \\
Understanding the overhead of different synchronization mechanisms
Balancing thread safety with performance
			\item Testing Multithreaded Code: \\
Techniques for writing unit tests for concurrent code
Tools for detecting race conditions and deadlocks
			\item Distributed Systems Concepts: \\
While not directly related to this problem, understanding concepts like eventual consistency and distributed locking can be beneficial
			\item Algorithms for Concurrent Operations: \\
Compare-And-Swap (CAS) operations
Lock-free algorithms
			\item Memory Models:
				\begin{itemize}
					\item Understanding memory barriers and volatile variables
					\item Cache coherence issues in multi-core systems
				\end{itemize}
\item Practice Problems:
	\begin{itemize}
		\item Implement a thread-safe singleton
		\item Create a simple producer-consumer queue
		\item Implement a basic thread pool
		\item Solve classic concurrency problems like the dining philosophers problem
		\item Write a few test cases in addition to the solution
		\item Remember, for interviews, it's not just about knowing the solutions, but also being able to explain your reasoning, discuss trade-offs, and analyze the performance and correctness of your solutions.
		\item Lastly, be prepared to write code on a whiteboard or in a simple text editor. Practice implementing these concepts without relying on an IDE's features.
	\end{itemize}
\end{itemize}
\end{itemize}

\subsection{Behavioral implementation}
\subsection{Technical implementation}
\end{document}
\enddocument

List out all of the major problems I've had to overcome.
Be able to talk about each of those stories
PROBLEM
ACTION STEPS to solve
OUTCOME

for phone interview




