\documentclass{article}

\usepackage[a4paper,margin=1in]{geometry}
\usepackage[T1]{fontenc}
\usepackage{xcolor}
\usepackage{tgcursor}
\usepackage{listings}
\usepackage{hyperref}

\lstdefinestyle{cpp}{
    language=C++,
    basicstyle=\ttfamily\small,
    keywordstyle=\color{blue},
    stringstyle=\color{red},
    commentstyle=\color{gray},
    morecomment=[l][\color{magenta}]{\#},
    numbers=left,
    numberstyle=\tiny\color{gray},
    stepnumber=1,
    numbersep=8pt,
    showstringspaces=false,
    breaklines=true,
    frame=single,
    rulecolor=\color{black},
    backgroundcolor=\color{white},
    tabsize=2,
    captionpos=b
}


\begin{document}
\fontfamily{qcr}\selectfont
\title{Interview Review Chart}
\author{Karl Solomon}
\maketitle
\section{Parallelism}
    \subsection{Concepts}
      \begin{description}
        \item[Concurrency]
          When two or more tasks can start/run/complete in overlapping time periods. This does not necessarily mean they'll be running in overlapping time periods. Examples include: \\ RTOS
      \end{description}
      \begin{description}
        \item[Parallelism]
          When tasks run at the same time (e.g. on a multi-core CPU)
      \end{description}
      \begin{description}
        \item[Multithreading]
          When multiple tasks are running on a CPU. This can be implemented truly parallel where each task has access to separate HW/core. However, more common in desktop applications is SMT.
      \end{description}
      \begin{description}
        \item[SMT (Simultaneous Multithreading)]
          Multiple threads share 1 core. The thread instructions are pipelines s.t. they run mostly in-parallel, and when one is waiting for I/O the other can run uninhibited, however since only one thread can access a dedicated HW block at any given time they are not truly parallel.
      \end{description}
    \subsection{Implementation}
      \begin{itemize}
        \item Concurrency and Multithreading
          \begin{itemize}
            \item Basic concepts of threads and processes
            \item Thread synchronization mechanisms (mutexes, semaphores, locks)
            \item Race conditions:
              \begin{enumerate}
                \item Multiple threads accessing a shared resource
                \item At least one thread writes to the resource
                \item Lack of proper synchronization
              \end{enumerate}
            \item deadlocks
              \begin{lstlisting}[style=cpp]
std::mutex m1, m2;
// Must have 2 locks where they are locked in different orders in different locations/threads
thread1: m1.lock(); m2.lock(); m1.unlock(); m2.unlock();
thread2: m2.lock(); m1.lock(); m2.unlock(); m1.unlock();
              \end{lstlisting}
            \item Atomic operations \\
              \begin{lstlisting}[style=cpp]
std::atomic<T> var; // where T is a primitive type
var = val;
var.load(val);
var.store(val);
var.wait(); // waits until the value changes
T current = var.exchange(val); // writes val to var and gets previous/current value of var
// Compare-and-swap. Atomically compares object.value with that of expected. If bitwise-equal then replaces former with desired. Otherwise loads actual value into expected (via load operation)
bool res = var.compare_exchange_strong(expected, desired); // preferred when don't expect high contention and cost of retrying is significant. Simpler, but slower.
bool res = var.compare_exchange_weak(expected, desired);  // preferred if you're anyways retrying in a loop or cost of retrying is low. More efficient, but more complex.
              \end{lstlisting}
              \begin{lstlisting}[style=cpp]
// Weak usage
std::atomic<int> value{0};
int expected = 0;
int desired = 1;
while(!value.compare_exchange_weak(expected, desired)){
  ; // handle spurrious failures
}

// Strong usage
if(value.compare_exchange_strong(expected, desired)){
  // op successful
}
else
{
  // op failed
}
              \end{lstlisting}

          \end{itemize}
        \item Thread-Safe Data Structures:
          \begin{itemize}
            \item Concurrent collections
            \item Concurrent collections (e.g., ConcurrentQueue, ConcurrentBag)
            \item Lock-free data structures
            \item Understanding the differences between thread-safe and non-thread-safe collections
          \end{itemize}
        \item Design Patterns for Concurrency:
          \begin{itemize}
            \item Producer-Consumer pattern
            \item Readers-Writer pattern
            \item Thread pool pattern
          \end{itemize}
        \item Language-Specific Concurrency Features for C++:
          \begin{itemize}
            \item std::thread
            \item <atomic>
          \end{itemize}
        \item Callback Mechanisms:
          \begin{itemize}
            \item Function pointers
              \lstinputlisting[style=cpp] {sources/functionPointers.cpp}
            \item Delegates (in languages that support them)
            \item Lambda expressions
          \end{itemize}
        \item Performance Considerations:
          \begin{itemize}
            \item Understanding the overhead of different synchronization mechanisms \\
              Generally best practice to measure performance in single-threaded vs multithreaded/parallel environments. Since there is overhead with creating/cleaning up threads/processes it can make your program run slower in smaller data sets.
            \item Balancing thread safety with performance
          \end{itemize}
        \item Testing Multithreaded Code: \\
          \begin{itemize}
            \item Techniques for writing unit tests for concurrent code
            \item Tools for detecting race conditions and deadlocks
              \begin{itemize}
                \item \textbf{Helgrind}: Part of Valgrind suite. Checks for race conditions, but slow.
                \item \textbf{ThreadSanitizer}: Compiler flag in llvm/clang. Faster than Helgrind.
                \item \textbf{RacerD}: Meta's C++-specific concurrent static analyzer. Good for larget code-bases.
                \item \textbf{Clang Static Analyzer}: Detects some simple conditions.
              \end{itemize}
          \end{itemize}
        \item Distributed Systems Concepts: \\
          While not directly related to this problem, understanding concepts like eventual consistency and distributed locking can be beneficial
        \item Algorithms for Concurrent Operations: \\
          \begin{itemize}
            \item Compare-And-Swap (CAS) operations
            \item Lock-free algorithms
          \end{itemize}
        \item Memory Models:
          \begin{itemize}
            \item Understanding memory barriers and volatile variables
            \item Cache coherence issues in multi-core systems
              \begin{description}
                \item[Cache Coherence]
                  The process of ensuring that data is stored in multiple caches within a multiprocessor system is consistent and synchronized.
              \end{description}
              This ensures that all processors have a \textit{consistent} view of shared memory. Cache coherence protocols manage the flow of data between caches, updating cache lines and tracking the status of shared data. This can be complicated because it requires balancing perforamnce and coherence overhead. The 2 main types of protocols are Directory-based and Snoop-based.
              \begin{description}
                \item[Directory-based]
                  The sharing status of a block of physical memory is kept in just one location (the directory). The direcotry can also be distributed to improve scalability. Communication is established using point-to-point requests through the interconnection network.
              \end{description}
              \begin{description}
                \item[Snoop-based]
                  Every cache that has a copy of the data from a block of physical memory also has a copy of the sharing status of the block, but no centralized state is kept. Caches are all accessible via some broadcast medium (a bus or switch), and all cache controllers monitor or \textit{snoop} on the medium to determine whether or not they have a copy of a block that is requested on a bus or switch access. Requires broadcast, csince caching information is at processors. This is useful for small-scale machines.
              \end{description}
              \begin{description}
                \item[Point of Coherency (PoC)]
                  Point at which all agents in a system which can access memory are guaranteed to see the same data.
              \end{description}
              \begin{description}
                \item[Migration]
                  Data is migrated to the local cache levels.
              \end{description}
              \begin{description}
                \item[Replication]
                  The same data is replicated across all caches.
              \end{description}
              Assume Snoop-based protocol. There are 2 ways to maintaint coherence:
              \begin{enumerate}
                \item \textbf{Write Invalidate Protocol}: Ensure that a processor has exclusive access to a data item before it writes that item. This is most common protocol.
                \item \textbf{Write Broadcast/Update}: All cached copies are updated simultaneously. This requires more bandwidth. When multiple updates happen to the same location, unnecessary updates are done. However, this is a lower latency between write/read.
              \end{enumerate}
          \end{itemize}
        \item Practice Problems:
          \begin{itemize}
            \item Implement a thread-safe singleton
              \lstinputlisting[style=cpp] {sources/threadSafeSingleton.cpp}
            \item Create a simple producer-consumer queue
              \lstinputlisting[style=cpp] {sources/producerConsumerQueue.cpp}
            \item Implement a basic thread pool
              \lstinputlisting[style=cpp] {sources/basicThreadPool.cpp}
            \item Solve classic concurrency problems like the dining philosophers problem
              \lstinputlisting[style=cpp] {sources/diningPhilosophers.cpp}
          \end{itemize}
      \end{itemize}
      Write a few test cases in addition to the solution
      Remember, for interviews, it's not just about knowing the solutions, but also being able to explain your reasoning, discuss trade-offs, and analyze the performance and correctness of your solutions.
      Lastly, be prepared to write code on a whiteboard or in a simple text editor. Practice implementing these concepts without relying on an IDE's features.

\section{Operating Systems}
    \subsection{Caches}
      \begin{description}
        \item[TLB]
          The TLB is a small, fast, and fast-access memory that is used to translate virtual memory addresses into physical memory addresses. a.k.a. Translation Lookaside Buffer.
      \end{description}
    \subsection{Basic Concepts}
      \begin{itemize}
        \item \textbf{TLB: Translate Lookaside Buffer}
        \item \textbf{Processes and Threads}
          \begin{itemize}
            \item Process creation and termination
            \item Thread lifecycle and management
          \end{itemize}
        \item \textbf{Memory Management}
          \begin{itemize}
            \item Virtual memory
            \item Paging and segmentation
          \end{itemize}
        \item \textbf{File Systems}
          \begin{itemize}
            \item File system structure
            \item File operations and permissions
          \end{itemize}
      \end{itemize}


\section{C}
    \subsection{Preprocessor}
      \begin{lstlisting}[style=cpp]
# // stringizes the macro parameter
#define stringify(x) #x
#define foo 1
stringify(foo) // --> evaluates to "foo", NOT "1"
\end{lstlisting}

      \begin{lstlisting}[style=cpp]
## // concatenates the macro parameter
#define COMMAND(NAME)  {#NAME, NAME ## _command}
struct command commands[] = {
  COMMAND(quit), // equivalent to {quit_command}
  COMMAND(help), // equivalent to {help_command}
}
\end{lstlisting}

      \begin{itemize}
        \item predefined macros
          \begin{itemize}
            \item \_\_FILE\_\_
            \item \_\_LINE\_\_
            \item \_\_DATE\_\_
            \item \_\_TIME\_\_
            \item \_\_STDC\_VERSION\_\_
            \item \_\_cplusplus
          \end{itemize}
        \item item2
        \item item3
        \item item4
      \end{itemize}

    \subsection{Peripherals}
      \begin{itemize}
        \item \textbf{I2C}\\
          SDA is data, SCL is clock. PURs typically in the 1-4.7k range. Too weak = slow comm and errors. Clocks are usually 100k-1MHz. Addr can be 7 or 10 bit. This is rate-limiter for number of slaves, though line impedance would increase for each slave. Here are some use usage examples:
          \begin{enumerate}
            \item Master sends START and slave Addr
            \item Master sends data to slave
            \item Master terminates with a STOP
          \end{enumerate}
          \begin{enumerate}
            \item Master sends START and slave Addr
            \item Master sends data to slave
            \item Master sends repeatedSTART and either sends more data to slave or receives data from slave.
            \item Master sends STOP
          \end{enumerate}
        \item \textbf{SPI}\\
          Serial Peripheral Interface (SPI) is a synchronous serial communication protocol used for short-distance communication, primarily in embedded systems. It uses a master-slave architecture with a single master and multiple slaves. Communication is full-duplex, and it requires four wires: MOSI, MISO, SCLK, and SS.
        \item \textbf{UART}\\
          Universal Asynchronous Receiver-Transmitter (UART) is a hardware communication protocol that uses asynchronous serial communication with configurable speed. It is commonly used for communication between microcontrollers and peripherals. UART requires only two wires: TX (transmit) and RX (receive).
        \item \textbf{USB}\\
          Universal Serial Bus (USB) is an industry-standard for short-distance digital data communications. It supports plug-and-play installation and hot swapping. USB is used for connecting peripherals such as keyboards, mice, printers, and external storage devices to computers.
        \item \textbf{HDMI}\\
          High-Definition Multimedia Interface (HDMI) is a proprietary audio/video interface for transmitting uncompressed video data and compressed or uncompressed digital audio data from an HDMI-compliant source device to a compatible display device. It is commonly used for connecting devices like TVs, monitors, and projectors.
      \end{itemize}
\section{C++}
    \subsection{<algorithm>}
      \begin{itemize}
        \item \textbf{batchOperations}
          \begin{itemize}
            \item for\_each, ranges::for\_each, for\_each\_n, ranges::for\_each\_n \\
              \lstinputlisting[style=cpp] {sources/foreach.cpp}
          \end{itemize}
        \item \textbf{Search Operations}
          \begin{itemize}
            \item all\_of, any\_of, none\_of \\
              \lstinputlisting[style=cpp] {sources/all_of.cpp}
            \item ranges::contains, ranges::contains\_subrange \\
              \lstinputlisting[style=cpp] {sources/contains.cpp}
            \item find, find\_if, find\_if\_not, ranges::find, ranges::find\_if, ranges::find\_if\_not \\
              \lstinputlisting[style=cpp] {sources/find.cpp}
            \item find\_last, find\_last\_if, find\_last\_if\_not, find\_end, ranges::find\_end \\
              \lstinputlisting[style=cpp] {sources/find_last.cpp}
          \end{itemize}
        \item \textbf{find\_end, ranges::find\_end, find\_first\_of, ranges::find\_first\_of}
        \item \textbf{adjacent\_find, ranges::adjacent\_find}
        \item \textbf{count, count\_if, ranges::count, ranges::count\_if}
        \item \textbf{mismatch, ranges::mismatch, equal, ranges::equal}
        \item \textbf{search, search\_n, ranges::search, ranges::search\_n}
        \item \textbf{ranges::starts\_with, ranges::ends\_with}
        \item \textbf{}
        \item \textbf{}
        \item \textbf{}
        \item \textbf{}
        \item \textbf{}
        \item \textbf{}
        \item \textbf{}
        \item \textbf{}
        \item \textbf{}
        \item \textbf{}
        \item \textbf{}
        \item \textbf{}
        \item \textbf{}
        \item \textbf{}
        \item \textbf{std::transform} \\
          \lstinputlisting[style=cpp] {sources/transform.cpp}
        \item \textbf{item2}
        \item \textbf{item3}
        \item \textbf{item4}
      \end{itemize}
    \subsection{Classes}
      \begin{itemize}
        \item \textbf{Class Definition}
          \begin{itemize}
            \item Syntax and structure
            \item Access specifiers: public, private, protected
          \end{itemize}
        \item \textbf{Inheritance}
          \begin{itemize}
            \item Single and multiple inheritance
            \item Virtual inheritance
          \end{itemize}
        \item \textbf{Polymorphism}
          \begin{itemize}
            \item Function overloading
            \item Operator overloading
            \item Virtual functions and abstract classes
          \end{itemize}
      \end{itemize}
    \subsection{Containers}
      \begin{itemize}
        \item Sequence
          \begin{itemize}
            \item \textbf{array}
              \begin{lstlisting}[style=cpp]
std::array<int, 3> arr; // uninitialized (whatever was in memory before)
std::array<int, 3> arr = {}; // initialized as 0s
std::array<int, 3> arr1 = {1, 2, 3};
std::array<int, 3> arr2{1, 2, 4};
arr1.fill(0); // fills array with 0s
arr1.swap(arr2); // swaps contents of arr1 and arr2
      \end{lstlisting}
            \item \textbf{vector}
              \begin{lstlisting}[style=cpp]
std::vector<int> v;
v.capacity(); // size of currently allocated memory
v.shrink_to_fit(); // releases unused memory
v.reserve(100); // pre-allocates 100 elements
v.clear(); // erases all elements
v.erase(v.begin()); // erases first element
v.push_back(1); // adds 1 to the end
v.rbegin(); // reverses iterator
std::erase_if(v, [](int x) { return x > 10; }); // removes all elements > 10
std::vector<Pair<int,int>> classV;
classV.emplace_back(10,1); // create Pair object and push to back
         \end{lstlisting}
            \item \textbf{inplace\_vector}
            \item \textbf{deque}
          \end{itemize}
        \item Associative
          \begin{itemize}
            \item \textbf{Set}
            \item \textbf{Map}
            \item \textbf{Multiset}
            \item \textbf{Multimap}
          \end{itemize}
        \item Unordered Associative
          \begin{itemize}
            \item \textbf{unordered\_set}
            \item \textbf{unordered\_map}
            \item \textbf{unordered\_multiset}
            \item \textbf{unordered\_multimap}
          \end{itemize}
        \item Adaptors
          \begin{itemize}
            \item \textbf{stack}
            \item \textbf{queue}
            \item \textbf{priority\_queue}
            \item \textbf{flat\_set}
            \item \textbf{flat\_map}
            \item \textbf{flat\_multiset}
            \item \textbf{flat\_multimap}
          \end{itemize}
      \end{itemize}
    \subsection{Modern C++}
      \begin{itemize}
        \item C++11
          \begin{itemize}
            \item \textbf{Alias Templates}
              \lstinputlisting[style=cpp] {sources/aliasTemplates.cpp}
            \item \textbf{atomic} \\
              Well-defined behavior in the event of RMW race contition. Accesses to atomics may establish inter-thread synchronization and order non-atomic accesses.
              \begin{lstlisting}[style=cpp]
atomic_bool b; // same as std::atomic<bool> b;
         \end{lstlisting}
            \item \textbf{auto}
            \item \textbf{constexpr}
            \item \textbf{final} \\
              \begin{itemize}
                \item Specifies that a class cannot be inherited from.
                \item When used in a virtual function, specifies that the function cannot be overridden by a derived class.
                \item final is also a legal variable/function name. Only has special meaning in member function declaration or class head.
              \end{itemize}
              \begin{lstlisting}[style=cpp]
struct Base
{
    virtual void foo();
};
struct A : Base
{
    void foo() final; // Base::foo is overridden and A::foo is the final override
    void bar() final; // Error: bar cannot be final as it is non-virtual
};
 
struct B final : A // struct B is final
{
    void foo() override; // Error: foo cannot be overridden as it is final in A
};
 
struct C : B {}; // Error: B is final
         \end{lstlisting}
            \item \textbf{initializer list}
              \lstinputlisting[style=cpp] {sources/initializerList.cpp}
            \item \textbf{iota}
              \begin{lstlisting}[style=cpp]
void iota(ForwardIterator begin, ForwardIterator end, T v); // fills range [first-last] with sequentially increasing values starting at v in begin
         \end{lstlisting}
            \item \textbf{lambdas}
              \lstinputlisting[style=cpp] {sources/lambdas.cpp}
              \begin{description}
                \item[capture]
                  comma-separated list of variables which are captured/modified by the lambda. Captures cannot have same name as input parameters.
              \end{description}
              Capture list
              \begin{itemize}
                \item \verb!&! = capture all used variables by reference
                \item \verb!=! = capture all used variables by copy
                \item \verb!varName! = by-copy
                \item \verb!varName...! = by-copy pack-expansion
                \item \verb!varName initializer! = by-copy  w/ initializer
                \item \verb!&varName! = by-reference
                \item \verb!&varName...! = by-reference pack-expansion
                \item \verb!&varName initializer! = by-reference w/ initializer
                \item \verb!this! = by-reference capture of current object
                \item \verb!*this! = by-copy capture of current object
                \item ... = by-copy capture of all objects w/ pack expansion
                \item \verb!&... initializer = by-reference w/ initializer and pack expansion!
              \end{itemize}
              \begin{lstlisting}[style=cpp]
// If the capture-default is &, subsequent simple captures must not begin with &. 
[&] {};          // OK: by-reference capture default
[&, i] {};       // OK: by-reference capture, except i is captured by copy
[&, &i] {};      // Error: by-reference capture when by-reference is the default
[&, this] {};    // OK, equivalent to [&]
[&, this, i] {}; // OK, equivalent to [&, i]
         \end{lstlisting}
              \begin{lstlisting}[style=cpp]
// If the capture-default is =, subsequent simple captures must begin with & or be *this(since C++17) or this(since C++20). 
[=] {};        // OK: by-copy capture default
[=, &i] {};    // OK: by-copy capture, except i is captured by reference
[=, *this] {}; // until C++17: Error: invalid syntax
               // since C++17: OK: captures the enclosing S2 by copy
[=, this] {};  // until C++20: Error: this when = is the default
                   // since C++20: OK, same as [=]
         \end{lstlisting}
              \begin{lstlisting}[style=cpp]
         \end{lstlisting}
            \item \textbf{mutex}
            \item \textbf{override}
            \item \textbf{random}
              \begin{lstlisting}[style=cpp]
#include <stdlib>
int rand(); // returns integer in [0, RAND_MAX]
         \end{lstlisting}
              \begin{lstlisting}[style=cpp]
#include <random>
// default_random_engine
// philox4x64 -> philox_engine
// random_device = non-deterministic generator based on hardware entropy
std::random_device rd;
rd.entropy(); // estimate of random number device entropy. Deterministic entropy = 0.
std::uniform_real_distribution<double> dist(0.0, 1.0);
         \end{lstlisting}
              Distribution list
              \begin{itemize}
                \item uniform
                  \begin{itemize}
                    \item int
                    \item real (double)
                  \end{itemize}
                \item bernoulli
                  \begin{itemize}
                    \item bernoulli
                    \item binomial
                    \item negative binomial
                    \item geometric
                  \end{itemize}
                \item Poisson
                  \begin{itemize}
                    \item poisson
                    \item exponential
                    \item gamma
                    \item weibull
                    \item extreme\_value
                  \end{itemize}
                \item Normal
                  \begin{itemize}
                    \item normal
                    \item lognormal
                    \item chi\_squared
                    \item cauchy
                    \item fisher\_f
                    \item student\_t
                  \end{itemize}
                \item Sampling
                  \begin{itemize}
                    \item discrete
                    \item piecewise\_constant
                    \item piecewise\_linear
                    \item item4
                  \end{itemize}
              \end{itemize}
            \item \textbf{range-based for}
            \item \textbf{thread}
            \item \textbf{trailing return type}
              \verb!auto main() --> int {return 0;}!
          \end{itemize}
        \item C++14
          \begin{itemize}
            \item \textbf{Variable Templates}
            \item \textbf{Generic Lambdas}
          \end{itemize}
        \item C++17
          \begin{itemize}
            \item \textbf{tuple}
              \lstinputlisting[style=cpp] {sources/tuples.cpp}
            \item \textbf{execution policies}
              \begin{description}
                \item[seq]
                  used to disambiguate parallel algorithm overloading and require that a parallel algorithm's execution must be sequential. This is used by default when no execution policy is specified.
              \end{description}
              \begin{description}
                \item[par]
                  Indicates that a parallel algorithm MAY be parallelized. Synchronization techniques (e.g. mutexes) may be used.
              \end{description}
              \begin{description}
                \item[par\_unseq]
                  A parallel algoithm MAY be parallelized, vectorized, and moved between threads. Vectorization MUST not use any vectorization-unsafe operations (e.g. mutexes and std::atomic)
              \end{description}
              \begin{description}
                \item[unseq]
                  An algorithm's execution MAY be vectorized. Synchronization techniques MUST NOT be used. Since C++20 (the rest of the policies were introduced in C++17).
              \end{description}
          \end{itemize}
        \item C++20
          \begin{itemize}
            \item \textbf{Modules}
            \item \textbf{Coroutines}
            \item \textbf{Ranges} \\
              Extension/Generalization of algorithms and iterator libraries to make them less error-prone. Ranges are an abstraction of the following:
              \begin{itemize}
                \item \verb![begin, end) iterator pair : ranges::sort()!
                \item \verb!begin + [0, size) : views::counted()!
                \item \verb![begin, predicate) : views::take\_while() (conditionally-terminated sequences)!
                \item \verb![begin, ..) : unbounded (e.g. views::iota())!
              \end{itemize}
              \verb!std::views // shorthand for std::ranges::views!
              TODO: do more usage/investigation on these


            \item \textbf{Midpoint} \\
              Can be used on any arithmetic type, excluding bool. Can be used on objects as long as they are not incomplete types. Returns half the sum of the two inputs, no overflow occurs (this is the main reason to use STL rather than custom implementation). Inputs must point to elements in same object, else behavior is undefined. In case of decimal in average, rounds down.

            \item \textbf{using enum}
            \item \textbf{constinit}
            \item \textbf{string formatting}
            \item \textbf{template concepts}
            \item \textbf{coroutines}
            \item \textbf{modules}
          \end{itemize}
        \item C++23
          \begin{itemize}
            \item \textbf{print/println}
              \begin{lstlisting}[style=cpp]
#include <print>
std::print("{0} {2}{1}!", "Hello", 23, "C++"););
std::println(); // adds newline to std::print();
         \end{lstlisting}
            \item \textbf{byteswap}
              \begin{lstlisting}[style=cpp]
#include <bit>
std::byteswap(T n) noexcept; // T can be any integer value
       \end{lstlisting}
            \item \textbf{flat\_map/flat\_set}
          \end{itemize}
      \end{itemize}
    \subsection{Concepts}
      \begin{itemize}
        \item \textbf{Types}
        \item \textbf{RAII}
          \begin{description}
            \item[RAII]
              Resource Acquisition Is Initialization
          \end{description}
        \item \textbf{item3}
        \item \textbf{item4}
      \end{itemize}

\end{document}
\enddocument
