\documentclass{article}

\usepackage[a4paper,margin=1in]{geometry}
\usepackage[T1]{fontenc}
\usepackage{xcolor}
\usepackage{tgcursor}
\usepackage{listings}

\lstdefinestyle{cpp}{
    language=C++,
    basicstyle=\ttfamily\small,
    keywordstyle=\color{blue},
    stringstyle=\color{red},
    commentstyle=\color{gray},
    morecomment=[l][\color{magenta}]{\#},
    numbers=left,
    numberstyle=\tiny\color{gray},
    stepnumber=1,
    numbersep=8pt,
    showstringspaces=false,
    breaklines=true,
    frame=single,
    rulecolor=\color{black},
    backgroundcolor=\color{white},
    tabsize=2,
    captionpos=b
}


\begin{document}
\fontfamily{qcr}\selectfont
\title{Interview Review Chart}
\author{Karl Solomon}
\maketitle
\section{Concepts}
\subsection{Concurrency vs Parallelism}
\begin{description}
	\item[Concurrency]
	      When two or more tasks can start/run/complete in overlapping time periods. This does not necessarily mean they'll be running in overlapping time periods. Examples include: \\ RTOS
\end{description}
\begin{description}
	\item[Parallelism]
	      When tasks run at the same time (e.g. on a multi-core CPU)
\end{description}
\begin{description}
	\item[Multithreading]
	      When multiple tasks are running on a CPU. This can be implemented truly parallel where each task has access to separate HW/core. However, more common in desktop applications is SMT.
\end{description}
\begin{description}
	\item[SMT (Simultaneous Multithreading)]
	      Multiple threads share 1 core. The thread instructions are pipelines s.t. they run mostly in-parallel, and when one is waiting for I/O the other can run uninhibited, however since only one thread can access a dedicated HW block at any given time they are not truly parallel.
\end{description}
\section{Operating Systems}
\section{C}
\subsection{Preprocessor}
\begin{lstlisting}[style=cpp]
# // stringizes the macro parameter
#define stringify(x) #x
#define foo 1
stringify(foo) // --> evaluates to "foo", NOT "1"
\end{lstlisting}

\begin{lstlisting}[style=cpp]
## // concatenates the macro parameter
#define COMMAND(NAME)  {#NAME, NAME ## _command}
struct command commands[] = {
	COMMAND(quit), // equivalent to {quit_command}
	COMMAND(help), // equivalent to {help_command}
}
\end{lstlisting}

\begin{itemize}
	\item predefined macros
	      \begin{itemize}
		      \item \_\_FILE\_\_
		      \item \_\_LINE\_\_
		      \item \_\_DATE\_\_
		      \item \_\_TIME\_\_
		      \item \_\_STDC\_VERSION\_\_
		      \item \_\_cplusplus
	      \end{itemize}
	\item item2
	\item item3
	\item item4
\end{itemize}

\subsection{Peripherals}
\begin{itemize}
	\item \textbf{I2C}\\
	      SDA is data, SCL is clock. PURs typically in the 1-4.7k range. Too weak = slow comm and errors. Clocks are usually 100k-1MHz. Addr can be 7 or 10 bit. This is rate-limiter for number of slaves, though line impedance would increase for each slave. Here are some use usage examples:
	      \begin{enumerate}
		      \item Master sends START and slave Addr
		      \item Master sends data to slave
		      \item Master terminates with a STOP
	      \end{enumerate}
	      \begin{enumerate}
		      \item Master sends START and slave Addr
		      \item Master sends data to slave
		      \item Master sends repeatedSTART and either sends more data to slave or receives data from slave.
		      \item Master sends STOP
	      \end{enumerate}
	\item \textbf{SPI}\\
	\item \textbf{UART}\\
	\item \textbf{USB}\\
	\item \textbf{HDMI}\\
\end{itemize}
\section{C++}
\subsection{Classes}
\subsection{Containers}
\begin{itemize}
	\item Sequence
	      \begin{itemize}
		      \item \textbf{array}
		            \begin{lstlisting}[style=cpp]
std::array<int, 3> arr; // uninitialized (whatever was in memory before)
std::array<int, 3> arr = {}; // initialized as 0s
std::array<int, 3> arr1 = {1, 2, 3};
std::array<int, 3> arr2{1, 2, 4};
arr1.fill(0); // fills array with 0s
arr1.swap(arr2); // swaps contents of arr1 and arr2
				  \end{lstlisting}
		      \item \textbf{vector}
		            \begin{lstlisting}[style=cpp]
std::vector<int> v;
v.capacity(); // size of currently allocated memory
v.shrink_to_fit(); // releases unused memory
v.reserve(100); // pre-allocates 100 elements
v.clear(); // erases all elements
v.erase(v.begin()); // erases first element
v.push_back(1); // adds 1 to the end
v.rbegin(); // reverses iterator
std::erase_if(v, [](int x) { return x > 10; }); // removes all elements > 10
std::vector<Pair<int,int>> classV;
classV.emplace_back(10,1); // create Pair object and push to back
			      \end{lstlisting}
		      \item \textbf{inplace\_vector}
		      \item \textbf{deque}
	      \end{itemize}
	\item Associative
	      \begin{itemize}
		      \item \textbf{Set}
		      \item \textbf{Map}
		      \item \textbf{Multiset}
		      \item \textbf{Multimap}
	      \end{itemize}
	\item Unordered Associative
	      \begin{itemize}
		      \item \textbf{unordered\_set}
		      \item \textbf{unordered\_map}
		      \item \textbf{unordered\_multiset}
		      \item \textbf{unordered\_multimap}
	      \end{itemize}
	\item Adaptors
	      \begin{itemize}
		      \item \textbf{stack}
		      \item \textbf{queue}
		      \item \textbf{priority\_queue}
		      \item \textbf{flat\_set}
		      \item \textbf{flat\_map}
		      \item \textbf{flat\_multiset}
		      \item \textbf{flat\_multimap}
	      \end{itemize}
\end{itemize}
\subsection{Modern C++}
\begin{itemize}
	\item C++11
	      \begin{itemize}
		      \item \textbf{Alias Templates}
		      \item \textbf{atomic} \\
		            Well-defined behavior in the event of RMW race contition. Accesses to atomics may establish inter-thread synchronization and order non-atomic accesses.
		            \begin{lstlisting}[style=cpp]
			      atomic_bool b; // same as std::atomic<bool> b;
			      \end{lstlisting}
		      \item \textbf{auto}
		      \item \textbf{constexpr}
		      \item \textbf{final} \\
		            \begin{itemize}
			            \item Specifies that a class cannot be inherited from.
			            \item When used in a virtual function, specifies that the function cannot be overridden by a derived class.
			            \item final is also a legal variable/function name. Only has special meaning in member function declaration or class head.
		            \end{itemize}
		            \begin{lstlisting}[style=cpp]
struct Base
{
    virtual void foo();
};
struct A : Base
{
    void foo() final; // Base::foo is overridden and A::foo is the final override
    void bar() final; // Error: bar cannot be final as it is non-virtual
};
 
struct B final : A // struct B is final
{
    void foo() override; // Error: foo cannot be overridden as it is final in A
};
 
struct C : B {}; // Error: B is final
			      \end{lstlisting}
		      \item \textbf{initializer list}
		      \item \textbf{iota}
		            \begin{lstlisting}[style=cpp]
void iota(ForwardIterator begin, ForwardIterator end, T v); // fills range [first-last] with sequentially increasing values starting at v in begin
			      \end{lstlisting}
		      \item \textbf{lambdas}
		            \begin{description}
			            \item[capture]
			                  comma-separated list of variables which are captured/modified by the lambda. Captures cannot have same name as input parameters.
		            \end{description}
		            Capture list
		            \begin{itemize}
			            \item \verb!&! = capture all used variables by reference
			            \item \verb!=! = capture all used variables by copy
			            \item \verb!varName! = by-copy
			            \item \verb!varName...! = by-copy pack-expansion
			            \item \verb!varName initializer! = by-copy  w/ initializer
			            \item \verb!&varName! = by-reference
			            \item \verb!&varName...! = by-reference pack-expansion
			            \item \verb!&varName initializer! = by-reference w/ initializer
			            \item \verb!this! = by-reference capture of current object
			            \item \verb!*this! = by-copy capture of current object
			            \item ... = by-copy capture of all objects w/ pack expansion
			            \item &... initializer = by-reference w/ initializer and pack expansion
		            \end{itemize}
		            \begin{lstlisting}[style=cpp]
// If the capture-default is &, subsequent simple captures must not begin with &. 
[&] {};          // OK: by-reference capture default
[&, i] {};       // OK: by-reference capture, except i is captured by copy
[&, &i] {};      // Error: by-reference capture when by-reference is the default
[&, this] {};    // OK, equivalent to [&]
[&, this, i] {}; // OK, equivalent to [&, i]
			      \end{lstlisting}
		            \begin{lstlisting}[style=cpp]
// If the capture-default is =, subsequent simple captures must begin with & or be *this(since C++17) or this(since C++20). 
[=] {};        // OK: by-copy capture default
[=, &i] {};    // OK: by-copy capture, except i is captured by reference
[=, *this] {}; // until C++17: Error: invalid syntax
               // since C++17: OK: captures the enclosing S2 by copy
[=, this] {};  // until C++20: Error: this when = is the default
                   // since C++20: OK, same as [=]
			      \end{lstlisting}
		            \begin{lstlisting}[style=cpp]
			      \end{lstlisting}
		      \item \textbf{mutex}
		      \item \textbf{override}
		      \item \textbf{random}
		            \begin{lstlisting}[style=cpp]
			      #include <stdlib>
			      int rand(); // returns integer in [0, RAND_MAX]
			      \end{lstlisting}
		            \begin{lstlisting}[style=cpp]
			      #include <random>
			      // default_random_engine
			      // philox4x64 -> philox_engine
			      // random_device = non-deterministic generator based on hardware entropy
			      std::random_device rd;
			      rd.entropy(); // estimate of random number device entropy. Deterministic entropy = 0.
			      std::uniform_real_distribution<double> dist(0.0, 1.0);
			      \end{lstlisting}
		            Distribution list
		            \begin{itemize}
			            \item uniform
			                  \begin{itemize}
				                  \item int
				                  \item real (double)
			                  \end{itemize}
			            \item bernoulli
			                  \begin{itemize}
				                  \item bernoulli
				                  \item binomial
				                  \item negative binomial
				                  \item geometric
			                  \end{itemize}
			            \item Poisson
			                  \begin{itemize}
				                  \item poisson
				                  \item exponential
				                  \item gamma
				                  \item weibull
				                  \item extreme\_value
			                  \end{itemize}
			            \item Normal
			                  \begin{itemize}
				                  \item normal
				                  \item lognormal
				                  \item chi\_squared
				                  \item cauchy
				                  \item fisher\_f
				                  \item student\_t
			                  \end{itemize}
			            \item Sampling
			                  \begin{itemize}
				                  \item discrete
				                  \item piecewise\_constant
				                  \item piecewise\_linear
				                  \item item4
			                  \end{itemize}
		            \end{itemize}
		      \item \textbf{range-based for}
		      \item \textbf{thread}
		      \item \textbf{trailing return type}
	      \end{itemize}
	\item C++14
	      \begin{itemize}
		      \item \textbf{Variable Templates}
		      \item \textbf{Generic Lambdas}
	      \end{itemize}
	\item C++17
	      \begin{itemize}
		      \item \textbf{tuple}
		      \item \textbf{execution policies}
	      \end{itemize}
	\item C++20
	      \begin{itemize}
		      \item \textbf{Modules}
		      \item \textbf{Coroutines}
		      \item \textbf{Ranges}
		      \item \textbf{Midpoint}
		      \item \textbf{using enum}
		      \item \textbf{constinit}
		      \item \textbf{string formatting}
		      \item \textbf{template concepts}
	      \end{itemize}
	\item C++23
	      \begin{itemize}
		      \item \textbf{print/println}
		      \item \textbf{byteswap}
		      \item \textbf{flat\_map/flat\_set}
	      \end{itemize}
\end{itemize}
\subsection{C++23}

\end{document}
\enddocument
